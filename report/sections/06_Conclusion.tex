\section{Conclusion}

%This project provides a first insight into the differences of free and paid articles on the platform Medium.com. We build a custom dataset consisting of a wide sample of articles published since 2020 and evalutated different aspects of the data. Firstly through statistical methods, embedding-based topic model and finally an analysis of the text written. 
In all aspects we find significant differences between the two groups.

Reader on Medium.com engage more with paid articles, indicated by the increased mean clap count of 407 versus 122 and an average response count of 4.9 comments per articles compared to 1.2. 
While the differences in article length are smaller, all are proven to be statistically significant given our large sample size.

Topic modeling reveales a strong thematic difference between paid and free articles. 
Technical domains, particulary Crypto \& Web3, is dominated by free content driven by enthusiastic open-knowledge interested writers. 
In contrast, personal and lifestyle topics such as Health \& Well-being and Relationship Dynamics are heavily skewed towards paid content, indicating the monetarization aspect of the authors. Cluster-level hypothesis tests strongly reject a uniform distribution, confirming the differences inside the clusters are significant.

Text quality analysis results show smaller differences. 
Paid articles contain siginificantly fewer grammar errors on average (12.8 vs 16.8), though this effect is small, suggesting that the difference to the reader is hard to notice. 
Further breakdown reveales that free articles carry more mechanical errors, while paid ones have more stylistic repetitions. 
In terms of AI detection, no significant differences exist, yet showing close aglignment with the LLM-release schedule.

To answer the initial question, whether a subscription is worth purchasing, it depends on the buyer's interests. For readers with special interest in personal topics, a membership can be very interesting, while programmers benefit less from the paid articles.